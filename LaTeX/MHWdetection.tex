\documentclass[utf8]{frontiersSCNS}

\usepackage{url,hyperref,lineno,microtype,subcaption,gensymb,booktabs}
\usepackage[onehalfspacing]{setspace}

\linenumbers

\def\keyFont{\fontsize{8}{11}\helveticabold }
\def\firstAuthorLast{Sample {et~al.}} %use et al only if is more than 1 author
\def\Authors{Robert W. Schlegel\,$^{1,*}$, Eric C. J. Oliver\,$^{2}$, Alistair J. Hobday\,$^{3}$ and Albertus J. Smit\,$^{1}$}
% Affiliations should be keyed to the author's name with superscript numbers and be listed as follows: Laboratory, Institute, Department, Organization, City, State abbreviation (USA, Canada, Australia), and Country (without detailed address information such as city zip codes or street names).
% If one of the authors has a change of address, list the new address below the correspondence details using a superscript symbol and use the same symbol to indicate the author in the author list.
\def\Address{$^{1}$Department of Biodiversity and Conservation Biology, University of the Western Cape, Bellville, South Africa \\
$^{2}$Department of Oceanography, Dalhousie University, Halifax, Nova Scotia, Canada \\
$^{3}$Oceans and Atmosphere, CSIRO, Hobart, Tasmania, Australia}
% $^{x}$Laboratory X, Institute X, Department X, Organization X, City X , State XX (only USA, Canada and Australia), Country X  }
% The Corresponding Author should be marked with an asterisk
% Provide the exact contact address (this time including street name and city zip code) and email of the corresponding author
\def\corrAuthor{Department of Biodiversity and Conservation Biology, University of the Western Cape, Private Bag X17, Bellville 7535, South Africa}

\def\corrEmail{robwschlegel@gmail.com}




\begin{document}
\onecolumn
\firstpage{1}

\title[MHW Detection]{Detecting marine heatwaves}

\author[\firstAuthorLast ]{\Authors} %This field will be automatically populated
\address{} %This field will be automatically populated
\correspondance{} %This field will be automatically populated

\extraAuth{}% If there are more than 1 corresponding author, comment this line and uncomment the next one.
%\extraAuth{corresponding Author2 \\ Laboratory X2, Institute X2, Department X2, Organization X2, Street X2, City X2 , State XX2 (only USA, Canada and Australia), Zip Code2, X2 Country X2, email2@uni2.edu}


\maketitle


\begin{abstract}
It is now known that marine heatwaves (MHWs) have been increasing in duration and intensity globally. It is therefore necessary that the detection of these events be possible in all areas of the ocean, inlcluding those with, for any number of reasons, sub-optimal access to long (>30 year) time series. This includes but is not limited to time series that are shorter than the proscribed 30 years, missing large amounts of random or consistent data, or have been collected with instrumentation from which the appropriate meta-data were not maintained. The best practices for how to deal with these issues have been investigated and outlined in detail here.

\tiny
 \keyFont{ \section{Keywords:} marine heatwaves, ocean, remotely sensed data, reanalysis data, \emph{in situ} data, climate change, code:R, code:Python}
 %All article types: you may provide up to 8 keywords; at least 5 are mandatory.
\end{abstract}


\section{Introduction}


\section{Material and Methods}


\subsection{Study region}


\subsection{Data}


\subsubsection{Remotely sensed data}


\subsubsection{Reanalysis data}


\subsubsection{\emph{In situ} data}


\subsection{Marine heatwaves (MHWs)}


\section{Results}


\section{Discussion}


\section{Conclusions}


\section*{Conflict of Interest Statement}
%All financial, commercial or other relationships that might be perceived by the academic community as representing a potential conflict of interest must be disclosed. If no such relationship exists, authors will be asked to confirm the following statement:
The authors declare that the research was conducted in the absence of any commercial or financial relationships that could be construed as a potential conflict of interest.


\section*{Author Contributions}


\section*{Funding}
This research was supported by NRF Grant number CPRR14072378735.


\section*{Acknowledgements}


\bibliographystyle{frontiersinSCNS_ENG_HUMS}
\bibliography{MHWdection}

%\section*{Figure captions}

% \begin{figure}[]
% \begin{center}
% \includegraphics[width=1.0\textwidth]{figure_1.pdf}
% \end{center}
% \caption{}
% \label{figure1}
% \end{figure}


% \section*{Table captions}

% \begin{table}[ht]
% \caption{The descriptions for the metrics of MHWs as proposed by \citet{Hobday2016} and adapted from \citet{Schlegel2017}.}
% \label{table1}
% \centering
% \tiny
% \begin{tabular}{ll}
% \toprule
%  Name [unit] & Definition \\
%  \midrule
%   Count [no. events per year] & \emph{n}: number of MHWs per year \\
%   Duration [days] & \emph{D}: consecutive period of time that temperature exceeds the threshold \\
%   Maximum intensity [\degree C] & \emph{i\textsubscript{max}}: highest temperature anomaly value during the MHW \\
%   Mean intensity [\degree C] & \emph{i\textsubscript{mean}}: mean temperature anomaly during the MHW \\
%   Cumulative intensity [\degree C$\cdot$days] & \emph{i\textsubscript{cum}}: sum of daily intensity anomalies over the duration of the event \\
%   Onset rate [\degree C$/$day] & \emph{r\textsubscript{onset}}: daily increase from event onset to maximum intensity \\
%   Decline rate [\degree C$/$day] & \emph{r\textsubscript{decline}}: daily decrease from maximum intensity to event end \\
%   \bottomrule
%   \end{tabular}
% \end{table}

\end{document}
